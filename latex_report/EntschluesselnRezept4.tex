\subsection{Rezept 4 entschlüsseln}
\label{RezeptVierEntschluesseln}

Eine Häufigkeitsanalyse ergab, dass die Verteilung der deutschen Sprache
entsprach und somit eine Transpositionschiffre vorliegen musste. Der Aufbau des
Geheimtextes legte zudem nahe, dass der Algorithmus zeilen- bzw. absatzweise
angewendet wurde, da jede Zeile des Geheimtextes, die keine Leerzeile war, mit
einem Großbuchstaben oder einer Ziffer begann, jedoch nie mit einem
Kleinbuchstaben, also das erste Zeichen einer Klartextzeile seine Position
behält.

Unter diesen Voraussetzungen kann eine Entschlüsselung bzw. eine Analyse zur
Feststellung des Verschlüsselungsverfahrens zunächst anhand einer beliebigen
Zeile unabhängig von den anderen Zeilen vorgenommen werden. Die erste Zeile des
Geheimtextes (``Sarcwrbohzt'') war die kürzeste und bedeutete als Klartext wohl
``Schwarzbrot''.

Auf Basis dieser Vermutung wurde zunächst eine passende Permutation einer
bestimmten Länge gesucht, die sequentiell auf die Zeile bis zur ihrem Ende
anzuwenden wäre. Allerdings zeigte sich bei Anwendung aller passenden (d.h. den
Klartext ergebenden) Permutationen auf die nächste Textzeile, dass kein
sinnvoller Klartext entstand. Nun hätte es sein können, dass es daran lag, dass
die Länge der korrekten Permutation die Länge der ersten Zeile (11 Zeichen)
überstieg. Allerdings war wahrscheinlich ein anderer Algorithmus verwendet
worden, da die Anzahl möglicher Permutationen mit der Fakultät ihrer Länge
zunehmen, aber die gegebene Aufgabenstellen als lösbar anzunehmen war.

Die Ermittlung des Algorithmus würde sehr erleichtert werden, wenn mehr Klartext
zur Verfügung stand. Unter der Annahme, dass beide Rezepte des Webservers aus
einer gemeinsamen Internetquelle stammen, wurde daher zunächst ein Ausschnitt
aus dem anderen, bereits entschlüsselten Rezept gegoogelt. Auf der
entsprechenden Rezepte-Homepage wurde anschließend eine Suche nach
Schwarzbrot-Rezepten durchgeführt. Unter den angezeigten Treffern wurde wiederum
dasjenige Rezept herausgesucht, dessen Absätze mit denjenigen Buchstaben des
verschlüsselten Schwarzbrot-Rezepts begannen - zur Erinnerung: es lag ja die
Vermutung nahe, dass der erste Buchstabe jedes Absatzes des Klartextes im
Geheimtext seine Position beibehielt, da es keine Kleinbuchstaben am
Zeilenbeginn gab.

Es fand sich tatsächlich ein passendes Rezept, welches für die weitere Analyse
herangezogen wurde. Durch Überlegen und manuelles Probieren wurde schließlich
folgender Verschlüsselungsalgorithmus identifiziert: Jede Zeile des Klartextes
wird in Vierergruppen aufgeteilt, die man untereinander schreibt. Der Geheimtext
ergibt sich, indem man die Zeichen der ersten der entstandenen Spalten
hintereinander schreibt, dann zeilenweise die Zeichen der zweiten und vierten
Spalte und erst komplett danach die Zeichen der dritten Spalte.

Mit diesem Wissen wurde der Geheimtexts entschlüsselt. Der Klartext war nicht
völlig mit dem Rezept auf der Internetseite identisch; beispielsweise war der
Position der Zutaten im Rezept verändert und einige Zeichen waren ausgetauscht
worden, es war etwa das „ß“ durch „ss“ ersetzt worden.
