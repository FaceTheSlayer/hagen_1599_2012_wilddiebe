\documentclass[%
	12t,			% Schriftgr��e
	a4paper,			% Papierformat
	oneside,			% einseitiges Dokument
	titlepage,			% es wird eine Titelseite verwendet
	%parskip=full,		% Abstand zwischen Abs�tzen (volle Zeile)=full halbe zeille = half
	headings=normal,	% Gr��e der �berschriften verkleinern
	DIVcalc,
	%DIV12,
	%BCOR=5mm,		% Bindekorrektur
	final
	,tablecaptionabove
]{scrartcl} 

%% Deutsche Anpassungen %%%%%%%%%%%%%%%%%%%%%%%%%%%%%%%%%%%%%
\usepackage[ngerman]{babel}
\usepackage[T1]{fontenc}
\usepackage[ansinew]{inputenc}

%\usepackage[
%	automark,		% Kapitelangaben in Kopfzeile automatisch erstellen
%	headsepline,		% Trennlinie unter Kopfzeile
%	ilines			% Trennlinie linksb�ndig ausrichten
%]{scrpage2}

\usepackage{lmodern} %Type1-Schriftart f�r nicht-englische Texte

%Formatierung geht ab hier los:
 
%Formatierung geht ab hier los:
\pagestyle{headings}
\setkomafont{sectioning}{\normalcolor\bfseries} % �ndert �berschriftschriftart (nun mit Serifen)

\usepackage{setspace}
\onehalfspacing																	%Zeilenabstand 1,5 fach
%\KOMAoptions{DIV=last}		%Satzspiegel neu berechnen wegen 1,5 fach Zeilenabstand
\typearea[current]{calc}

% Kopf- und Fu�zeilen ------------------------------------------------------
%\pagestyle{scrheadings}
 
%\usepackage{textcomp}
%\usepackage{marvosym}
%\usepackage{mathrsfs}

\usepackage{amsmath}

\usepackage[activate]{pdfcprot}

\usepackage{graphicx} % jpg  png  pdf  mps

\usepackage{xcolor}

\usepackage{pbox}

\definecolor{darkblue}{rgb}{0,0,.6}

\usepackage{url}


%\usepackage[square, numbers]{natbib}


\usepackage{listings}

\usepackage[german]{varioref}

\usepackage[
pdftitle={Fachpraktikum 01599: IT-Sicherheit},
pdfsubject={Wilddiebe 3},
pdfauthor={Stefan G�tz, J�rgen Klemm, Thorsten Klingert, Thomas Koch, Gerald Luber, Gisela Nagy, Stefan T�uber},
pdfkeywords={},
colorlinks=true,
linkcolor=darkblue,
citecolor=darkblue,
filecolor=darkblue,
urlcolor=darkblue,
bookmarks=true,
bookmarksopen=true,
bookmarksopenlevel=3,
plainpages=false,
pdfpagelabels=true]{hyperref}

\addto\captionsngerman{\renewcommand{\refname}{Literaturverzeichnis}}

\pdfcompresslevel=1

%\usepackage{relsize}
\usepackage{float}

%\usepackage{enumitem}

\usepackage{booktabs}

\usepackage{bibgerm}

\makeatletter
\renewcommand{\@biblabel}[1]{[#1]\hfill}
\makeatother

\makeatletter
\renewcommand\paragraph{%
   \@startsection{paragraph}{4}{0mm}%
      {-\baselineskip}%
      {.5\baselineskip}%
      {\normalfont\normalsize\bfseries}}
\makeatother

\newcommand{\Csharp}{%
\textsf{C\kern-.09em\raise.40ex\hbox{\smaller{\#}}}}

\newcommand{\randnotiz}[1]{%
\emph{#1}\marginpar{#1}}

\newcommand{\bilddatei}[3]{%
\begin{figure}[H]
 \centering
 \includegraphics[width=#3\textwidth,keepaspectratio=true]{Images/#1}
 \caption{#2}
 \label{fig:#1}
\end{figure}}

\newcommand{\bildref}[1]{%
\cref{fig:#1}}

\newcommand{\starttable}[3]{%
\begin{table}[H]
\caption{#2}
\label{tab:#1}
\centering
\begin{tabular}{#3}}

\newcommand{\stoptable}{%
\end{tabular}
\end{table}}

\newcommand{\tableref}[1]{%
\cref{tab:#1}}

\definecolor{bluekeywords}{rgb}{0.13,0.13,1}
\definecolor{greencomments}{rgb}{0,0.5,0}
\definecolor{redstrings}{rgb}{0.9,0,0}

\lstset{language=[Sharp]C,
frameround=single,
showspaces=false,
showtabs=false,
breaklines=true,
showstringspaces=true,
breakatwhitespace=true,
escapeinside={(*@}{@*)},
commentstyle=\color{greencomments},
keywordstyle=\color{bluekeywords}\bfseries,
stringstyle=\color{redstrings},
basicstyle=\ttfamily
}

\lstnewenvironment{CSharpCode}[2]
  {\lstset{language=[Sharp]C,caption=#2,label=#1,captionpos=b, tabsize=2}}
  {}
  
\lstdefinelanguage{Metasploit}
{keywords=
{openssl},%
sensitive=false,%
alsoletter={\$},%
comment=[l]{\#},%
string=[b]",%
string=[b]'%
commentstyle=\color{greencomments},%
keywordstyle=\color{bluekeywords}\bfseries,%
stringstyle=\color{redstrings},%
basicstyle=\ttfamily%
}

%\lstnewenvironment{MetasploitCode}[2]
%  {\lstset{language=Metasploit,caption=#2,label=#1,captionpos=b, tabsize=2}}
%  {}

\lstnewenvironment{MetasploitCode}[2]
  {\lstset{language=Metasploit, tabsize=2}}
  {}

\usepackage{filecontents}

\bibliographystyle{alphadin}

\usepackage[german]{cleveref}

\setcounter{tocdepth}{4}  % = Aufnahme in das Inhaltsverzeichnis *
\setcounter{secnumdepth}{4}%   = Nummerierung vertiefen *

%%%%%%%%%%%%%%%%%%%%%%%%%%%

\newcommand{\Mayerbrot}{\emph{Mayer Brot}}
\newcommand{\Muellerbrot}{\emph{M�ller Brot}}
\newcommand{\Metasploit}{\emph{Metasploit}}
\newcommand{\Armitage}{\emph{Armitage}}
\newcommand{\Firefox}{\emph{Firefox}}
\newcommand{\zB}{z.\,B.}
\newcommand{\va}{v.\,a.}
\newcommand{\dH}{d.\,h.}

\hyphenation{Sys-tem-ad-mi-nis-tra-tor}
\hyphenation{So-cial}
\hyphenation{En-gi-nee-ring}