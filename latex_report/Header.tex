\documentclass[%
	12t,			% Schriftgröße
	titlepage,			% es wird eine Titelseite verwendet
	%parskip=full,		% Abstand zwischen Absätzen (volle Zeile)=full halbe zeille = half
	headings=normal,	% Größe der Überschriften verkleinern
        DIV=calc,
	%DIV12,
	%BCOR=5mm,		% Bindekorrektur
        captions=tableheading
]{scrartcl}

\KOMAoptions{
  paper=a4,
  twoside=false,  % replaces standard option oneside
  draft=false     % replaces standard option final
}

%% Deutsche Anpassungen %%%%%%%%%%%%%%%%%%%%%%%%%%%%%%%%%%%%%
\usepackage[ngerman]{babel}
\usepackage[T1]{fontenc}
\usepackage[utf8]{inputenc}
\usepackage{lmodern} %Type1-Schriftart für nicht-englische Texte
\usepackage{amsmath}
\usepackage[activate]{pdfcprot}
\usepackage{graphicx} % jpg  png  pdf  mps
\usepackage{xcolor}
\usepackage{pbox}
%\usepackage{url}
\usepackage{listings}
\usepackage[german]{varioref}
\usepackage{float}
\usepackage{booktabs}
\usepackage{bibgerm}
\usepackage{filecontents}
\usepackage[german]{cleveref}
\usepackage[
pdftitle={Fachpraktikum 01599: IT-Sicherheit},
pdfsubject={Wilddiebe 3},
pdfauthor={Stefan Götz, Jürgen Klemm, Thorsten Klingert, Thomas Koch, Gerald Luber, Gisela Nagy, Stefan Täuber},
pdfkeywords={},
colorlinks=true,
linkcolor=darkblue,
citecolor=darkblue,
filecolor=darkblue,
urlcolor=darkblue,
bookmarks=true,
bookmarksopen=true,
bookmarksopenlevel=3,
plainpages=false,
pdfpagelabels=true]{hyperref}

%Formatierung geht ab hier los:
\pagestyle{headings}
\setkomafont{sectioning}{\normalcolor\bfseries} % Ändert Überschriftschriftart (nun mit Serifen)

\usepackage{setspace}
\onehalfspacing			%Zeilenabstand 1,5 fach
\typearea[current]{calc}

\addto\captionsngerman{\renewcommand{\refname}{Literaturverzeichnis}}

\pdfcompresslevel=1

\makeatletter
\renewcommand{\@biblabel}[1]{[#1]\hfill}
\makeatother

\makeatletter
\renewcommand\paragraph{%
   \@startsection{paragraph}{4}{0mm}%
      {-\baselineskip}%
      {.5\baselineskip}%
      {\normalfont\normalsize\bfseries}}
\makeatother

\definecolor{darkblue}{rgb}{0,0,.6}
\definecolor{bluekeywords}{rgb}{0.13,0.13,1}
\definecolor{greencomments}{rgb}{0,0.5,0}
\definecolor{redstrings}{rgb}{0.9,0,0}

\lstset{language=[Sharp]C,
frameround=single,
showspaces=false,
showtabs=false,
breaklines=true,
showstringspaces=true,
breakatwhitespace=true,
escapeinside={(*@}{@*)},
commentstyle=\color{greencomments},
keywordstyle=\color{bluekeywords}\bfseries,
stringstyle=\color{redstrings},
basicstyle=\ttfamily
}

\lstdefinelanguage{Metasploit}%
{keywords={openssl},%
sensitive=false,%
alsoletter={\$},%
comment=[l]{\#},%
string=[b]",%
string=[b]'%
commentstyle=\color{greencomments},%
keywordstyle=\color{bluekeywords}\bfseries,%
stringstyle=\color{redstrings},%
basicstyle=\ttfamily%
}

\lstnewenvironment{MetasploitCode}[2]
  {\lstset{language=Metasploit, tabsize=2}}
  {}

\bibliographystyle{alphadin}

\setcounter{tocdepth}{4}  % = Aufnahme in das Inhaltsverzeichnis *
\setcounter{secnumdepth}{4}%   = Nummerierung vertiefen *

%%%%%%%%%%%%%%%%%%%%%%%%%%%

\newcommand{\bilddatei}[3]{%
\begin{figure}[H]
 \centering
 \includegraphics[width=#3\textwidth,keepaspectratio=true]{Images/#1}
 \caption{#2}
 \label{fig:#1}
\end{figure}}
\newcommand{\bildref}[1]{\cref{fig:#1}}

\newcommand{\tableref}[1]{\cref{tab:#1}}

\newcommand{\Mayerbrot}{\emph{Mayer Brot}}
\newcommand{\Muellerbrot}{\emph{Müller Brot}}
\newcommand{\Metasploit}{\emph{Metasploit}}
\newcommand{\Armitage}{\emph{Armitage}}
\newcommand{\Firefox}{\emph{Firefox}}
\newcommand{\zB}{z.\,B.}
\newcommand{\va}{v.\,a.}
\newcommand{\dH}{d.\,h.}

\hyphenation{Sys-tem-ad-mi-nis-tra-tor}
\hyphenation{So-cial}
\hyphenation{En-gi-nee-ring}
\hyphenation{Home-page}
