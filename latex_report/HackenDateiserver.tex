\subsection{Dateiserver hacken}
\label{sec:dateiserverhacken}

%\bilddatei{DateiServerHacken1}{Dienste}{0.7}
%
%\bilddatei{DateiServerHacken2}{Check Exploits}{0.7}
%
%\bilddatei{DateiServerHacken3}{Angriffsziel}{0.7}
%
%\bilddatei{DateiServerHacken4}{Erfolgreicher Angriff}{0.7}
%
%\bilddatei{DateiServerHacken5}{Dateien anzeigen}{0.7}
%
%\bilddatei{DateiServerHacken5}{Datei herunterladen}{0.7}

Um zu überprüfen, welche Dienste laufen, wird ein Portscanner für die Erkennung
benötigt.

\subsubsection{NMAP}

Nmap ist wohl der bekannteste Netzwerkscanner. Er spürt aktive Hosts im Netz auf
und nutzt eine breite Palette an Tests vom normalen TCP/IP-Handshake bis zum
verborgenen TCP-FIN-Scan. Aufgrund der Eigenheiten der TCP/IP-Stacks erkennt
nmap das Betriebssystem und Dienste.

Unter Free-BSD kann NMap mit folgender Befehlszeile installiert werden:

\begin{verbatim}
root:~$cd /usr/ports/net/nmap && make install clean
\end{verbatim}

\begin{figure}
\begin{lstlisting}
#!/bin/bash

DT=`/bin/date +\%H_\%M_\%S_\%d_\%m_\%y`
LOG_FILE=scan_${DT}.log

nmap -Avv 172.16.14.0/24 > $LOG_FILE
./nsscan.sh "$LOG_FILE"

\end{lstlisting}
\caption{Script für dynamischen Portscan: nmap.sh}
\end{figure}

\begin{figure}
\begin{lstlisting}
#!/bin/bash

ips=`cat $1 | grep "open port" | awk '{ print $6 }' | sort | uniq`

for i in $ips 
do
  nslookup $i 172.16.19.1 >> tmp 
done

cat tmp | grep name | sort | uniq > $1_names.txt

rm -f tmp
\end{lstlisting}
\caption{Script zur Namensauflösung: nsscan.sh}
\end{figure}

Mit Hilfe beider des Skriptes nmap.sh (das nsscan.sh intern aufruft) erhält man
eine Liste von Hosts und ihrer offenen Ports im Netzwerk. Die Hosts werden
zunächst als IP Adressen aufgeführt und darunter werden die dazugehörigen DNS
Namen aufgelistet.
\begin{figure}
\begin{lstlisting}
Initiating Connect Scan at 19:27
Scanning 2 hosts [1000 ports/host]
Discovered open port 22/tcp on 172.16.14.10
Discovered open port 80/tcp on 172.16.14.10
Discovered open port 445/tcp on 172.16.14.40
Discovered open port 1025/tcp on 172.16.14.40
Discovered open port 139/tcp on 172.16.14.40
Discovered open port 3389/tcp on 172.16.14.40
Discovered open port 135/tcp on 172.16.14.40
Discovered open port 1026/tcp on 172.16.14.40
.
.
.
10.14.16.172.in-addr.arpa	name = rootca.mayerbrot.local.
40.14.16.172.in-addr.arpa	name = datei.mayerbrot.local.
.
.
.
\end{lstlisting}
\caption{Script für dynamischen Portscan: nmap.sh}
\end{figure}

\subsubsection{Zugriff auf den Dateiserver}

Aus den offenen Ports des Rechners \lstinline{datei.mayerbrot.local} kann man
schließen, dass es sich um einen Windows Server handelt.. Es wurde mit
\\172.16.14.40 versucht, sich die Freigaben anzuzeigen. Es war zu diesem
Zeitpunkt nur public zu sehen. Auf das public Verzeichnis kann von einem Linux
oder Free-BSD Rechner einfach zugegriffen werden. Dazu wird die Freigabe
``gemountet'' und die Dateien können mit regulären UNIX Befehlen angesehen und
kopiert werden:

\begin{lstlisting}
\% mkdir /mnt/cifs
\% sudo mount -t cifs //172.16.14.40/public /mnt/cifs
\% cd /mnt/cifs
\% ls
Rezept1NR.txt  testdatei.txt
\% cp -v Rezept1NR.txt /home/xxx
`Rezept1NR.txt' -> `/home/xxx/Rezept1NR.txt'
\end{lstlisting}

\subsubsection{Metasploit}

Das Metasploit-Projekt ist ein freies Open-Source-Projekt zur
Computersicherheit, das Informationen über Sicherheitslücken bietet und bei
Penetrationstests sowie der Entwicklung von IDS-Signaturen(? @TODO) eingesetzt
werden kann. Das bekannteste Teilprojekt ist das Metasploit Framework, ein
Werkzeug zur Entwicklung und Ausführung von Exploits gegen verteilte
Zielrechner. Andere wichtige Teilprojekte sind das Shellcode-Archiv und
Forschung im Bereich der IT-Sicherheit.

\begin{lstlisting}

\end{lstlisting}
