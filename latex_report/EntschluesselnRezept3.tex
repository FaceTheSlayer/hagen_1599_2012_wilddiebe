\subsection{Rezept 3 entschlüsseln}
\label{RezeptDreiEntschluesseln}

Die verschlüsselte Datei wurde zunächste einer Häufigkeitsanalyse in
JCrypTool\footnote{Java Lehrsoftware für Kryptologie, erhältlich unter
  \url{http://www.cryptool.org}} unterzogen. Als Einstellungen wurden
``Monoalphabetisch'' sowie als Referenzverteilung ``DEUTSCH Descartes''
ausgewählt. Als ursprüngliches Alphabet des Klartextes wurde ``Printable ASCII''
genommen. Die Häufigkeitsanalyse zeigt \cref{fig:rezept3verteilung1}.

\bilddatei{rezept3verteilung1}{}{}

Durch einfaches Ziehen mit der Maus konnte man die tatsächliche Verteilung
(schwarz) mit der Referenzverteilung (weiß) in Übereinstimmung bringen. Dabei
wurde der Graph um 47 Stellen verschoben (\cref{fig:rezept3verteilung2}).

\bilddatei{rezept3verteilung2}{}{}

Mit Hilfe der Häufigkeitsverteilung kann man genau sehen, dass das kleine ``e''
in die Zahl ``6'' verschlüsselt wurde. Leider konnte mit den in JCrypTool
vorhandenen Alphabeten keine Entschlüsselung gelingen. So wurden zunächst
manuell mit Hilfe des Editors und Suchen und Ersetzen die Geheimschriftzeichen
durch die Klartextzeichen ersetzt. Es stellte sich dabei heraus, dass die
Umlaute und die Leerstellen nicht verschlüsselt wurden. Mit Hilfe eines eigens
angepaßten Alphabets konnte dann anschließend die Entschlüsselung auch mit dem
JCrypTool automatisiert werden. Das angepaßte Alphabet ist Printable ASCII ohne
Leerstellen und Umlaute (\cref{fig:rezept3alphabet}).

\bilddatei{rezept3alphabet}{}{}

Im Menü von JCrypTool unter Fenster $\rightarrow$ Benutzervorgaben $\rightarrow$
Kryptographie $\rightarrow$ Alphabete wurde das benutzerdefinierte Alphabet
hinzugefügt. Weiterhin wurde der Haken entfernt bei der Option ``Filtern von
ungültigen Zeichen''.  Mit Hilfe des angepaßten Alphabets und der Wahl von ``O''
als Schlüssel kann das Rezept nun ganz einfach entschlüsselt werden.  Der
angewandte Algorithmus ist Cesar-Chiffre mit einer Verschiebung um 47 Zeichen.
