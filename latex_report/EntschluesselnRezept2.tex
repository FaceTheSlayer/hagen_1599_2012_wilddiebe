\subsection{Rezept 2 entschlüsseln}
\label{RezeptZweiEntschluesseln}

\begin{lstlisting}[caption={Das verschlüsselte 2. Rezept: Rezept2NR.txt},
 linewidth=\textwidth,
 breaklines, breakatwhitespace=false]
HrSbkesnuz1eWbsgteeamhWbezldsiDeSgeraeuWgmneSnmmunnant
solsrivtieuMsusrnk3eieteean0vzBribAFhkau2gh15reailsron
n0eeeinceraa1zhsreboeonlnednnaudBciet0himareesbekqlnsn
WleicemareknzinanasegenisznaneDnetdsgnlnsarurieadc3teb
imeptreleDateSreaVkrhuaSfaMnneeeenftenteugabatomeacedS
dnZmtmauhnsnBcea2GCoiesraiMnnanuroembhlsZtn5ogele17get
WisuegmkeWe1Vlrwem5gnnmnn2ene2uhzngohcWle0l1leeWeietin
euereLimndnheni0lseserieadcullsDenstonesedckreesesarbs
eesueaebnSuegRgmhdaseteshbek1udeZeteaghasiSedaeidolnen
dazctuernnneettdihaettKefgbbenntnbimrprrelsDnkfu2rdren
aodn0uecesernealnsnae0RemTp51Suegznet30atas7gloezelSnb
ukr5Lsm5Bwi5gbekensga0misasdDnmneeewmmssgneeaiskdseuhs
WasNdteaioedsvcgtnirresenuglmdlhtkIgemlseneduueeegaeof
ahdt5tkdmneetgny0aier0ls5kiholegage0gtsS0ss
\end{lstlisting}

Da lediglich wieder nur der chiffrierte Text bekannt war, wurde Ciphertext-Only
gewählt.

\begin{table}\footnotesize
\begin{tabular}{*{26}{@{\hspace{1ex}}r}}
 0 & 1 & 2 & 3 & 5 & 7 & A & B & C & D & F & G & H & I & K & L & M & N & R & S & T & V & W & Z \\
 13 & 8 & 5 & 3 & 9 & 2 & 1 & 4 & 1 & 6 & 1 & 1 & 1 & 1 & 1 & 2 & 3 & 1 & 2 & 11 & 1 & 2 & 9 & 3 \\
\end{tabular}

\begin{tabular}{*{26}{@{\hspace{1ex}}r}}
 a & b & c & d & e & f & g & h & i & k & l & m & n & o & p & q & r & s & t & u & v & w & y & z \\
 50 & 17 & 12 & 26 & 134 & 5 & 29 & 19 & 33 & 16 & 27 & 26 & 75 & 16 & 3 & 1 & 34 & 52 & 34 & 28 & 3 & 3 & 1 & 10 \\
\end{tabular}
\end{table}

Die Häufigkeitsanalyse ergab wie auch beim 1. Rezept keine Abweichungen. Das
dechiffrieren des Textbeginns mit der Casesar Chiffre ergibt ein sinnloses
``EOPYHBPKRW1'', rot-13 ergibt ``UEFOXRFAHM1''.

\subsubsection{Rail Fence Cipher}

iese Chiffre nennt sich Rail Fence Cipher, aufgrund der Art wie der
Chiffriervorgang abläuft. Der Klartext wird diagonal nach rechts unten auf einen
imaginären Zaun geschrieben. Wenn das untere Ende erreicht wurde, wird diagonal
nach rechts oben bis zum oberen Ende geschrieben, usw. bis der gesamte Klartext
geschrieben ist.

Mit einer Tiefe von zwei entsteht aus dem Beginn von Rezept 2:

\begin{lstlisting}
H r S b k e s n u z 1 e W
 i e t i n e u e r e L i m
\end{lstlisting}

Mit einer Tiefe von drei entsteht

\begin{lstlisting}
H   r   S   b   k   e   s
 z n a n e D n e t d s g
  a   e   0   R   e   m
\end{lstlisting}
Und mit einer Tiefe von vier schließlich lässt sich ein Rezept erkennen:
\begin{lstlisting}
H     r     S     b     k
 e   b o   e o   n l   n
  i e   t i   n e   u e
   d     D     n     m
\end{lstlisting}



\begin{lstlisting}[caption={rafeci.c}]
#include <stdio.h>

char alphabet[1000] = {0};
char alphabet125[125] = {0};
char alphabet248_1[248] = {0};
char alphabet248_2[248] = {0};
char alphabet124[124] = {0};

const int LIM1 = 125;
const int LIM2 = 248;
const int LIM3 = 124;

int i = 0;
int i1 = 0;
int i2 = 0;

int main()
{
  int c;

  while ( ( c = getchar() ) != EOF ) {
    alphabet[i++] = c;
  }

  for (i2 = 0 ; i2 < LIM1; i2++ ) {
    alphabet125[i2] = alphabet[i1++];
  }

  for (i2 = 0 ; i2 < LIM2; i2++ ) {
    alphabet248_1[i2] = alphabet[i1++];
  }

  for ( i2 = 0; i2 < LIM2; i2++ ) {
    alphabet248_2[i2] = alphabet[i1++];
  }

  for ( i2 = 0; i2 < LIM3; i2++ ) {
    alphabet124[i2] = alphabet[i1++];
  }

  for ( i2 = 0; i2 < LIM1; i2++ ) {
    printf("\%c     ", alphabet125[i2]);
  }

  printf("\n");
  printf(" ");

  for ( i2 = 0; i2 < LIM2; i2++ ) {
    printf("\%c   ", alphabet248_1[i2]);
    printf("\%c ", alphabet248_1[++i2]);
  }

  printf("\n");
  printf("  ");

  for ( i2 = 0; i2 < LIM2; i2++ ) {
    printf("%c ", alphabet248_2[i2]);
    printf("%c   ", alphabet248_2[++i2]);
  }

  printf("\n");
  printf("   ");

  for ( i2 = 0; i2 < LIM3; i2++ ) {
    printf("%c     ", alphabet124[i2]);
  }

  return 0;
}
\end{lstlisting}

Das entschlüsselte Rezept lautet:

Heidebrot

Die Sonnenblumenkerne den Leinsamen und den Buchweizen mit 100ml heißem Wasser begießen, abgedeckt quellen lassen.
Die Walnüsse mit kochendem Wasser bedecken kurz ziehen lassen, das Wasser abgießen. Die Nüsse zu den Saaten geben. 
Den Sauerteig das Roggenmehl und das Wasser gut vermischen, abgedeckt 13 Stunden bei Zimmertemperatur gehen lassen.
Die Saatenden, Sauerteig das Vollkornmehl und das Salzfracht Minuten verkneten. In eine gefettete und mit Mehl
ausgestaubte Kastenform geben, abdecken und 3 Stunden bei Zimmertemperatur gehen lassen. Den Backofen auf 220 Grad C vorheizen
das Brot darin 50 Minuten backen Aus der Form nehmen abkühlen lassen. 

Zutaten 
250g Roggenmehl Type 1150, 175g Sauerteig Weizensauerteig, 300ml kaltes Wasser, 175g Vollkornweizenmehl, 50g 
Sonnenblumenkerne, 25g Leinsamen, 25g Buchweizen, 50g grobgehackte Walnüsse, 10g Salz, 100ml heißes Wasser
