\subsection{Webserver hacken}

Wir wissen bereits, dass der Webserver von \Mayerbrot{} unter der Adresse
172.16.14.30 zu erreichen ist\footnote{vgl. den Portscan aus
  \protect{\cref{fig:nmap}}}. Greifen wir über \Firefox{} auf diese Adresse zu,
dann werden wir auf \url{https://172.16.14.30} umgeleitet. Die Kommunikation
erfolgt also über eine gesicherte Verbindung. \Firefox{} kann das Zertifikat
nicht überprüfen, und gibt eine entsprechende Warnung aus. Wenn wir diese
ignorieren, gelangen wir zur Homepage von \Mayerbrot{} (siehe
\bildref{WebserverHacken2}).

\bilddatei{WebserverHacken2}{Homepage von \Mayerbrot{}}{0.7}

Das erste Rezept ist nicht weiter geschützt. Über den entsprechenden Link unter
\textsc{Neuigkeiten} kann das verschlüsselte Rezept heruntergeladen
werden\footnote{Die Entschlüsselung wird in \cref{RezeptDreiEntschluesseln}
  gezeigt}.

Auf der Homepage befindet sich auch ein \emph{Sicherheitsbereich}. Ein Zugriff
darauf wird mit einer Fehlermeldung quittiert (siehe
\bildref{WebserverHacken4}). Hier ist wohl das zweite Rezept zu finden. Wenn wir
ins Blaue hinein raten und \url{https://172.16.14.30/private/rezept.txt}
eintippen, kommen wir auch nicht weiter.

\bilddatei{WebserverHacken4}{Gesicherte Verbindung fehlgeschlagen}{0.7}

\subsubsection{Metasploit}

Nachdem \Metasploit{} bereits beim Dateiserver gute Dienste geleistet hat,
versuchen wir den Webserver auf ähnliche Weise zu attackieren. Durch den
vorausgegangenen Portscan haben wir schon eine Liste der
offenen Ports erhalten (siehe \cref{fig:offeneports}).

Um eventuelle Schwachstellen zu entdecken, starten wir \Armitage{}. Anschließend
fügen wir den Host 172.16.14.30 hinzu und führen einen Scan aus. Wenn dieser
abgeschlossen ist, können wir über \emph{\glqq{}Find Attacks\grqq{}}
Angriffspunkte suchen.

\Armitage{} findet Exploits in den Kategorien \texttt{http},
\texttt{realserver}, \texttt{webapp} und \texttt{wyse}. Mit \emph{\glqq{}Check
  Exploits\grqq{}} filtern wir erfolgsversprechende Exploits heraus. Als
einzigen Treffer erhalten wir \texttt{unix/webapp/basilic\_diff\_exec}
(vgl. \bildref{WebDatei1}). Wenden wir diesen Exploit an, führt das aber
irgendwie nicht zum gewünschten Erfolg.

\bilddatei{WebDatei1}{Webserver Exploit}{0.7}

In \Metasploit{} wird jedem Exploit ein Rang zugeordnet. Damit werden der
Wirkungsgrad und die Einfachheit des Exploits bewertet. Der Rang reicht von
\emph{Poor} bis \emph{Excellent}. Wir könnten jetzt unsere Ansprüche
herunterschrauben und auch schlechtere Exploits zulassen. Vielleicht sollten wir
unsere Suche aber auf andere Tools ausweiten.


\subsubsection{Webscanner}

Der Schwerpunkt von \Metasploit{} liegt nicht bei der Schwachstellen-Analyse von
Webapplikationen. Dafür existieren spezielle Webserver-Scanner, deren Ergebnisse
wiederum in \Metasploit{} importiert werden können. Beim Scannen von
Webapplikationen liefern kommerzielle Produkte wie \emph{Appscan},
\emph{Webinspect}, \emph{Acunetix} oder \emph{Burp} gute Ergebnisse. Frei
zugänglich sind dagegen \emph{Nikto}, \emph{W3AF}, \emph{Watobo}, \emph{Wapiti}
oder \emph{Nexpose Community Edition} (vgl. \cite[S.~281]{metahandbuch}). Wir
beginnen mit Nikto.

\paragraph{Nikto}

Bei Nikto handelt es sich um einen Webserver-Scanner, der auf den gängigen
Betriebssystemen (\emph{Windows}, \emph{Mac OSX}, \emph{Linux} und \emph{UNIX})
verfügbar ist. Nikto kann dabei über 6400 Probleme in CGI- und PHP-Dateien
erkennen. Außerdem wird nach veralteten Versionen oder bekannten Problemen
spezieller Versionen gesucht (vgl. \cite{Nikto}).

Gestartet wird die Analyse mit dem Befehl \texttt{nikto -h 172.16.14.30}. Als
Ergebnis werden folgende Warnungen erzeugt:

\begin{table}[H]
\caption{Nikto Warnungen}
\label{NiktoResult}
\centering
\begin{tabular}{cp{9cm}}
\toprule
OSVDB\footnotemark & Bemerkung\\
\midrule
27071 & PHP Image View 1.0 is vulnerable to Cross Site Scripting (XSS)\\
-- & Post Nuke 0.7.2.3-Phoenix is vulnerable to Cross Site Scripting (XSS) \\
4598 & Web Wiz Forums ver. 7.01 and below is vulnerable to Cross Site Scripting (XSS)\\
2946 & Web Wiz Forums ver. 7.01 and below is vulnerable to Cross Site Scripting (XSS)\\
2799 & DailyDose 1.1 is vulnerable to a directory traversal attack in the 'list' parameter\\
\bottomrule
\end{tabular}
\end{table}

\footnotetext{Abkürzung für \emph{Open Source Vulnerability Database} (siehe
  \url{http://www.osvdb.org})}

Auf der Webseite \url{http://www.metasploit.com/modules/} können wir prüfen, ob
\Metasploit{} entsprechende Exploits zur Verfügung stellt. Aber für keine
OSVDB-Nummer gibt es Treffer.
    
\paragraph{W3AF}

W3AF steht für \emph{Web Application Attack and Audit Framework}. Dabei handelt
es sich um ein Open-Source-Projekt, das bei \emph{SourceForge} gehostet wird.

Über Plugins wird eine erweiterbare Architektur bereitgestellt. Im Gegensatz zu
Nikto beschränkt sich W3AF dabei nicht nur auf die Schwachstellen-Analyse,
sondern bietet auch entsprechende Exploits an (vgl. \cite{W3AF}). W3AF kann
dabei bequem über eine graphische Oberfläche bedient werden.

Um den Webserver von \Mayerbrot{} zu scannen, geben wir als \emph{Target} die
Adresse \url{https://172.16.14.30} ein. Anschließend wählen wir die Plugins
\emph{Audit} und \emph{Discovery} (siehe \bildref{w3af1}). Mit \emph{Discovery}
veranlassen wir die Schwachstellen-Analyse und mit \emph{Audit} suchen wir
entsprechende Exploits. Danach starten wir den Scan-Vorgang.

\bilddatei{w3af1}{W3AF Scan config}{0.6}

Für die Analyse wird ein Proxy mit dem Namen \emph{spiderMan} gestartet, der
über \texttt{127.0.0.1:44444} zu erreichen sein wird. Wir konfigurieren \Firefox{}
so, dass die Kommunikation über diesen Proxy abläuft. Anschließend können wir
die Homepage von \Mayerbrot{} aufrufen und durch die Anwendung
navigieren. Sind wir damit fertig, dann geben wir die Adresse
\url{http://127.7.7.7/spiderMan?terminate} ein.

Die Prüfung hat einige Schwachstellen entdeckt, aber keine führt uns direkt zum
geheimen Rezept.

\bilddatei{w3af3}{W3AF Results}{0.6}

\subsubsection{Angriff über den Dateiserver}

Unsere bisherigen Bemühungen waren nicht von Erfolg gekrönt. Natürlich können
wir der Reihe nach alle Webserver-Scanner ausprobieren, in der Hoffnung, dass
einer etwas Brauchbares liefert. Vielleicht sollten wir aber neue Strategien ins
Auge fassen.

Wenn ein Einbrecher über das Schlafzimmerfenster in eine Wohnung eingebrochen
ist, wird er von dort aus die anderen Zimmer durchsuchen. Der Einbruch ins
Firmennetz von \Mayerbrot{} ist uns bereits über den Dateiserver gelungen. Wir
sollten uns fragen, ob wir diese Schwachstelle weiter ausnutzen
können. Zumindest dürfte der Dateiserver einen gewissen Vertrauensvorsprung
gegenüber firmenfremden Computern haben.

Um dies zu testen, übernehmen wir erneut den Dateiserver\footnote{dazu gehen wir
  wie in \cref{sec:dateiserverhacken} vor}. Maximale Kontrolle über den
Dateiserver erhalten wir, indem wir eine VNC-Sitzung aufbauen. Dazu gehen wir
über \emph{Meterpreter 1} zu \emph{Interact} und dann zum Eintrag \emph{Desktop
  (VNC)}. Anschließend steht der VNC-Dienst über \url{127.0.0.1:5934} zur
Verfügung. Wir tragen im VNC-Viewer von \Metasploit{} die angegebene Adresse
ein und erhalten Zugang zum Dateiserver (siehe \bildref{WebDatei12}).

\bilddatei{WebDatei12}{VNC-Sitzung mit Dateiserver}{0.6}

Auf dem Desktop ist das Icon von \emph{OpenVPN GUI} zu sehen. Aus unseren
Erfahrungen mit \emph{OpenVPN GUI}\footnote{vgl. \cref{sec:openvpn}} wissen wir,
dass im Konfigurationsverzeichnis auch das Zertifikat des Dateiservers zu
finden sein muss. Es kann nicht schaden, das komplette Konfigurationsverzeichnis auf
unseren Rechner zu kopieren.

Anschließend kümmern wir uns um das geheime Rezept. Dazu versuchen wir die
Homepage des Webservers über den Internet-Explorer zu erreichen. Leider
erscheint eine Fehlermeldung, dass die Datei nicht gefunden werden kann. Die
Seite von Google kann aber aufgerufen werden. Ein Ping auf den Webserver klappt
auch. Weitere Untersuchungen waren aber nicht möglich, da wir plötzlich nicht
mehr ungestört waren. Der Nachteil einer VNC-Sitzung ist, dass man nicht
unbeobachtet agieren kann. Als der Administrator sich auf System aufschaltet,
trennen wir daher die Verbindung. Nicht einmal die notwendigen Aufräumarbeiten,
um die Spuren des Einbruchs zu beseitigen, konnten wir durchführen.

\subsubsection{Zugriff mit dem Zertifikat}

Wie bereits erwähnt, haben wir vom Dateiserver die kompletten
OpenVPN-Konfigurationsdateien heruntergeladen
(vgl. \bildref{dateiserverdateien}). Darunter befindet sich auch das Zertifikat
des Dateiservers. Bei einem Zertifikat handelt sich um einen digitalen
Ausweis. Verwenden wir diesen Ausweis gegenüber dem Webserver, dann muss dieser
davon ausgehen, dass er es mit dem Dateiserver zu tun hat. Auf diese Weise
könnten wir unter dem Namen des Dateiservers ungestört weitere Tests
durchführen. Zunächst wollen wir prüfen, ob wir damit auf den geschützten Bereich
des Webservers zugreifen können.

\bilddatei{dateiserverdateien}{OpenVPN-Konfigurationsdateien des Dateiservers}{0.6}

Um ein Zertifikat in \Firefox{} zu importieren, muss es als PFX-Datei
vorliegen. Wir haben aber nur eine crt-Datei und den privaten
Schlüssel. Allerdings können wir daraus über folgendes Kommando eine PFX-Datei
erzeugen:

\begin{MetasploitCode}{lst:openssl}{openssl}
openssl pkcs12 -inkey privkey.pem -in datei.mayerbrot.local.vpn.crt -export -out mayerdatei.pfx
\end{MetasploitCode}

Bei der Frage nach dem Schlüssel wählen wir 123 und erhalten die Datei
\texttt{mayerdatei.pfx}. Dieser Schlüssel wird später benötigt, wenn wir das
Zertifikat in den Webbrowser importieren wollen.

Das machen wir auch sogleich. Dazu starten wir \Firefox{} und rufen den
Zertifikatsmanager auf. Hier gehen wir zur Registerkarte \glqq{}Ihre
Zertifikate\grqq{} und importieren die Datei \texttt{mayerdatei.pfx}. Als
Ergebnis erhalten wir das Fenster in \bildref{WebserverHacken8}.

\bilddatei{WebserverHacken8}{Importiertes Zertifikat}{0.6}

Wenn wir jetzt auf die Homepage des Webservers zugreifen wird zunächst nach dem
Zertifikat gefragt (vgl. \bildref{WebserverHacken9}). Hier wählen wir das
Zertifikat des Dateiservers.

\bilddatei{WebserverHacken9}{Zertifizierungsanfrage}{0.6}

Greifen wir jetzt auf den Sicherheitsbereich zu, dann erscheint die Meldung:
\glqq{}Dieser Text ist nur mit gültigem Client-Zertifikat abrufbar\grqq{}. So
schnell geben wir aber nicht auf und tippen
\url{https://172.16.14.30/private/rezept.txt} ein. Jetzt erhalten wir das
verschlüsselte Rezept aus dem Sicherheitsbereich\footnote{Die Entschlüsselung
  des Rezepts wird in \cref{RezeptVierEntschluesseln} gezeigt}
(vgl. \bildref{WebserverHacken11}). Damit sehen wir, dass es durchaus reicht,
einen Rechner zu hacken, um Zugriff auf weitere Rechner im Netz zu erhalten.

\bilddatei{WebserverHacken11}{Zweites Rezept des Webservers}{0.6}



