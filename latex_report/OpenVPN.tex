\section{OpenVPN}
\label{sec:openvpn}

OpenVPN ermöglicht die Einrichtung eines virtuellen privaten Netzwerkes (VPN),
das auf Secure Socket Layer (SSL) basiert. Die VPN-Tunnelendpunkte werden durch
virtuelle Schnittstellen realisiert. OpenVPN arbeitet nicht im Kernel-Space des
Betriebssystems, sondern im User-Space. Während eines SSL\hyp Handshakes werden
starke symmetrische Verschlüsselungsverfahren vereinbart, so dass die
Vertraulichkeit der Daten während der Übertragung durch das VPN gesichert
ist. Die Authentisierung erfolgt über X.509-Zertifikate oder durch vorab
vereinbarte Schlüssel.

Der VPN-Server kann ein einfacher Server auf PC-Hardware-Basis sein, auf dem
z.B. Linux läuft. Auf diesem Server installiert man die OpenVPN-Software.

Auf dem VPN-Client braucht man die OpenVPN-Client-Software. Auch sie ist frei
verfügbar und läuft unter Windows, Linux oder Mac OS X.

Die wechselseitige Authentisierung kann auf verschiedene Arten
erfolgen. Besonders einfach zu administrieren ist der VPN-Server, wenn
X.509-Zertifikate benutzt werden. Als Betreiber eines VPN-Servers betreibt man
dazu seine eigene Certification Authority (CA). Man erstellt ein
selbst-signiertes CA-Zertifikat und signiert damit dann das Zertifikat des
Servers und die Zertifikate der Benutzer.

Auf dem Server braucht man dann

\begin{itemize}
  \item das CA-Zertifikat,
  \item den privaten Schlüssel des Servers und
  \item das Server-Zertifikat.
\end{itemize}

Auf der Client-Seite braucht man

\begin{itemize}
  \item das CA-Zertifikat,
  \item den privaten Schlüssel des Clients und
  \item das Client-Zertifikat.
\end{itemize}

Nun kann der Client beim SSL-Handshake den Server anhand seines Zertifikates
authentisieren und der Server authentisiert den Client anhand des
Client-Zertifikates. Der VPN-Server braucht insbesondere keine Verbindung zu
einem Directory-Server oder Ähnliches. Er erkennt neue Benutzer (Clients) immer
daran, dass sie ein Zertifikat vorweisen, welches von der eigenen CA ausgestellt
wurde.

Die Konfiguration von OpenVPN ist sehr einfach. Auf der Client-Seite reicht
bereits eine wenige Zeilen lange Konfigurationsdatei. Die Einrichten des
OpenVPN\hyp Zugangs wird schrittweise am Beispiel von Windows XP erläutert.

\paragraph{Schritt 1}
Herunterladen der OpenVPN-Software (Windows Installer
\path{openvpn-2.2.2-install.exe}) von der Adresse
\url{http://openvpn.net/index.php/download/community-downloads.html} und
Installation dieser.

\paragraph{Erstellung eines eigenen Zertifikats}
Zur Erleichterung des Ablaufs und der Userinteraktion kommt ein von Michael
Franke von der Zertifikatserver-Gruppe Süd programmiertes Java-Applet zum
Einsatz. Dieses bietet zudem den sicherheitstechnischen Vorteil, dass der
private Schlüssel immer auf dem Rechner des Zertifikatbeantragers liegt und
nicht über das WWW oder andere Wege verbreitet werden muss.


\paragraph{Schritt 2}

Downloaden des von der Zertifikatserver-Gruppe Süd erstellten SSL-Tools von der
Adresse \url{http://major-bug.de/ssl_cert_request_generator.zip}

Die zip-Datei enthält unter anderem

\begin{itemize}
  \item den Zertifikatgenerator
  \item den öffentlichen Schlüssel \path{RootCA.muellerbackwaren.crt} des
    Zertifikatservers (notwendig, um das ausgestellte Zertifikat in eine
    Vertrauensstellung zu bringen)
  \item das Cross-Zertifikat \path{RootCA.mayerbrot.local.cross.crt} (notwendig, um
    auch im Netz des Fusionspartners sichere Verbindungen herzustellen)
  \item die Dateien \path{UserCA.muellerbackwaren.crt} und
    \path{ServerCA.muellerbackwaren.crt} (diese sind jeweils zweckgebunden zu
    verwenden)
\end{itemize}

\paragraph{Schritt 3}

Entpacken obiger zip-Datei und Ausführen des Generators
\path{ssl\_cert\_request\_generator.jar}.  Dazu ist es erforderlich, dass eine
aktuelle Java Ausführungsumgebung (JRE) auf dem Rechner installiert ist.

\paragraph{Schritt 4}

Eingabe der Benutzerdaten (Vorname, Name, Bundesland, Stadt,
Rücksende-E-Mail-Adresse), welche mit den Herrn Naues vorliegenden Personaldaten
übereinstimmen müssen.

\paragraph{Schritt 5}

Erzeugen des privaten Schlüssels durch einen Klick auf den Button \glqq{}Erzeugen\grqq{}.
Der Dateiname sollte die Endung *.key haben (z.B. \path{StefanTaeuberKey.key}).

\paragraph{Schritt 6}

Vergabe eines geheimen Passworts für den privaten Schlüssel.

\paragraph{Schritt 7}

Abschicken der (E-Mail mit der) Signierungsanfrage an die
Zertifikatserver-Gruppe Süd.

Das persönliche Zertifikat (mit dem signierten öffentlichen Schlüssel des
Zertifikatinhabers) bekommt man schließlich kurze Zeit später als E-Mail-Anhang.

\paragraph{Schritt 8 - Importieren der Zertifikate bei Windows XP}

Die Datei \path{RootCA.muellerbackwaren.crt} muss beim Import manuell in die
vertrauenswürdigen Stammzertifizierungsstellen eingefügt werden. Die anderen
Zertifikate können dann automatisch importiert werden.

\begin{itemize}
  \item \glqq{}Start\grqq{}
  \item \glqq{}Ausführen\grqq{}
  \item Eingabe von \path{certmgr.msc}
  \item \glqq{}OK\grqq{}
  \item \glqq{}Vertrauenswürdige Stammzertifizierungsstellen\grqq{} auswählen
  \item auf Objekttyp \glqq{}Zertifikate\grqq{} mit der rechten Maustaste klicken
  \item \glqq{}Alle Tasks\grqq{}
  \item \glqq{}Importieren\ldots\grqq{} 
  \item im Zertifikatsimport-Assistenten auf \glqq{}Weiter\grqq{} klicken
  \item Ort des Zertifikates \glqq{}(Durch)suchen\grqq{}
  \item Zertifikat auswählen und \glqq{}Öffnen\grqq{} anklicken
  \item \glqq{}Weiter\grqq{}
  \item untere Option \glqq{}Alle Zertifikate in folgendem Speicher speichern\grqq{} auswählen
  \item \glqq{}Weiter\grqq{}
  \item \glqq{}Fertigstellen\grqq{}
\end{itemize}

Auf gleiche Weise müssen auch die Zertifikate
\path{RootCA.mayerbrot.local.cross.crt}, \path{UserCA.muellerbackwaren.crt} und des
persönlichen Zertifikate \path{stefan.taeuber.muellerbackwaren.crt} importiert
werden.

\paragraph{Schritt 9 -- Konfiguration von OpenVPN}

In der zip-Datei der Zertifikatserver-Gruppe Süd (vgl. Schritt 2) ist die Datei
\path{muster1.conf} enthalten. Diese kann als Grundlage für die
OpenVPN\hyp Konfigurationsdatei dienen. Es müssen lediglich die Namen von privatem
und öffentlichem Schlüssel angepasst werden. Etwaige Probleme mit unsichtbaren
(Linux-)Steuerzeichen lassen sich z.B. dadurch beheben, dass man die
Konfigurationsparameter in einer \glqq{}frischen\grqq{} Textdatei manuell eintippt\ldots

Die Konfigurationsdatei \path{openvpn.config.ovpn} wird mit Hilfe eines
Text-Editors erstellt und enthält:

\begin{lstlisting}
client
float
dev tap
tun-mtu 1492
mssfix
proto udp
remote muellerbw.dyndns.org 11940
tls-remote server
ca ca-bundle.txt
cert stefan.taeuber.muellerbackwaren.crt
key StefanTaeuberKey.key
auth SHA1
nobind
comp-lzo
persist-key
persist-tun
verb 3
\end{lstlisting}

\paragraph{Schritt 10}

Verschieben der vier benötigten Dateien in den Dateiordner
\path{C:\Programme\OpenVPN\config}

\begin{enumerate}
  \item \path{ca-bundle.txt} (enthält die erforderlichen root- und intermediate-Zertifikate)
  \item \path{StefanTaeuberKey.key} (privater \glqq{}Wilddieb-Schlüssel\grqq{})
  \item \path{stefan.taeuber.muellerbackwaren.crt} (signiertes \glqq{}Wilddieb-Zertifikat\grqq{})
  \item \path{openvpn.config.ovpn} 
\end{enumerate}

\paragraph{Schritt 11 -- Einloggen ins Firmennetz mit OpenVPN}

Nachdem nun alle Vorarbeiten abgeschlossen sind, kann man sich mit Hilfe der
grafischen Bedienoberfläche OpenVPN GUI bequem ins Firmennetz einwählen.

\paragraph{Schritt 12}

Das entsprechende Symbol (zwei kleine rote Monitore) in der Startleiste mit der
linken Maustaste doppelt anklicken (alternativ: rechte Maustaste, danach
\glqq{}Connect\grqq{} auswählen), dann das in Schritt 6 vergebene geheime Passwort für den
privaten Schlüssel eintippen und mit \glqq{}OK\grqq{} bestätigen.

Die beiden kleinen Monitore des OpenVPN GUI-Symbols verfärben sich zunächst gelb
und wechseln schließlich nach erfolgreichem Verbindungsaufbau auf grün.

Es ist geschafft!

% Literaturliste
% Stefan Wohlfeil. Sicherheit im Internet 2. FernUniversität in Hagen, 10/2011.
% Zertifikatserver-Gruppe Süd. Handout_Muellerbackwaren.pdf, 06/2012.
